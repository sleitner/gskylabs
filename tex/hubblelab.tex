\documentclass[12pt]{article}
%\documentclass[12pt,preprint]{aastex}
%\usepackage{smsections}
\usepackage{graphicx}
\usepackage{amssymb}

\textwidth = 6.5 in
\textheight = 9 in
\oddsidemargin = 0.0 in
\evensidemargin = 0.0 in
\topmargin = -.3in
\headheight = 1ex
\headsep = 4ex
\parskip = 0.2in
\parindent = 0.0in
\title{Simple Lenses and Telescopes}
\begin{document}

\vspace*{-.8in}
\markright{\tiny \rm PS12000 W2009 V1.0}
\leftline{\tiny PS12000 W2009 V1.0}
\vspace{.5in}

\centerline{\large\bf Morphological types, spectra of galaxies,}
\centerline{\large\bf  and the Hubble expansion law.}
\vspace{.3in}

\begin{quote}
{\it "The history of astronomy is a history of receding horizons."}  --- Edwin Hubble
\end{quote}

\section{Introduction}
\pagestyle{myheadings}
\thispagestyle{plain}

In 1920s the University of Chicago alumnus Edwin Hubble\footnote{You can find detailed information
about life and work of Edwin Hubble on this website: {\tt http://www.time.com/time/time100/scientist/profile/hubble.html}} (Ph.D., 1917) solved four fundamental questions in astronomy and cosmology\footnote{See essay by Hubble's student, Alan Sandage at\\ {\tt http://antwrp.gsfc.nasa.gov/diamond$\_$jubilee/d$\_$1996/sandage$\_$hubble.html}}:
\begin{itemize}
\item 1922-1926. Hubble proposed a classification system for nebulae,
both galactic (which we still call ``nebulae'') and extragalactic
(which we know call ``galaxies''). The classification system of
galaxies has become known as the Hubble morphological sequence of galaxy types.
\item In 1924 Hubble decisively showed that spiral ``nebulae'' (or galaxies) are 
huge ``island universes'' of stars similar to our own Milky Way, 
with his discovery of Cepheids in nearby galaxies NGC 6822, M33 and M31 
(The Andromeda galaxy) which allowed to measure distances to them.

\item Having established the fact that galaxies are distant stellar systems
like the Milky Way, Hubble used their distribution in various areas on the sky
to show that the overall distribution of galaxies in space on large scales is
quite homogeneous. This was a crucial fact as it justified the cosmological 
principle of homogeneity and isotropy used to obtain dynamical models of 
expanding universe from the Einstein's equations of general relativity. 

\item In 1929 Hubble with a lot of technical help from his assistant Milton Humason has discovered 
linear relation between redshift of lines in galaxy spectra and their distance from the Milky Way: 
{\it the more distant a galaxy, the greater its
redshift.}
 This discovery lead to the notion of
the expanding universe and is the most fundamental discovery and observational fact that we know
about our observable Universe.
\end{itemize}
In this lab we will redo (in a simplified way) the Hubble's analysis
that resulted in the first, third, and fourth of these fundamental
discoveries using modern tools. The lab has four parts. In the first
part you will explore the morphological types of galaxies and the
overall shapes and features of their spectra, including the features
in the spectra called ``spectral lines.''  In the second part, you will
count galaxies in randomly chosen patches of the sky to see how homogeneous
their distribution over the sky really is. In the third part, you
will measure the wavelengths of spectral lines in real galaxy
spectra. To determine galaxy redshifts you will compare the measured
redshifted galaxy lines to the laboratory measurements of the line
wavelength on Earth.  Finally, using the apparent brightnesses of the
brightest cluster galaxies as proxy for their distances you will plot
your measured redshifts and distance against one another to produce
the Hubble law, which you will use to measure the Hubble constant.


\section*{Morphological types of galaxies}

Hubble classified galaxies into a sequence, now called {\it the tuning-fork Hubble diagram} (due
to the its ressemblance of a tuning fork viewed horizontally) based on their
visual appearance: shape, smoothness, and presence of features such as bars and spiral arms. All this
collectively is called ``galaxy morphology'' and classification scheme is called ``morphological 
classification.'' Hubble has conjectured
that the sequence represented the path of galaxy evolution. Although this was not confirmed
by further observations, the classification is still widely used by astronomers. 

Hubble's classification contains many sub-classes of galaxies\footnote{See the diagram at\\{\tt http://cas.sdss.org/dr3/en/proj/advanced/galaxies/tuningfork.asp}}, but the two main 
classes of galaxies are the elliptical and spiral galaxies. Ellipticals tend to be
smooth in their light distribution and roundish in their appearance. 
Spiral galaxies tend to have features such as bars and spiral arms. 

To familiarize yourself with main morphological types of galaxies, go to the Galaxy Zoo website
at \texttt{http://www.galaxyzoo.org/Tutorial.aspx} and review Part 1 of that page.

{\bf Lab tasks.} 
{\it
\begin{itemize}
\item What are the typical colors of elliptical and spiral galaxies? Discuss why these classes
of galaxies tend to have the colors they have from what you know about colors of stars. 
\item Sketch six examples of the ellipticals and six examples of spiral galaxies in your lab report indicating their
type (spiral or ellipticals) and providing your best guess for their sub-type according to the
Hubble's classification (e.g., E3, Sc, SBb, etc.). It may be useful to review this portion
of the Galaxy Zoo's website:\\ {\tt http://beta.galaxyzoo.org/howtotakepart.aspx} (see also
the corresponding chapter of your textbook). 
\end{itemize}
}

If you are curious and have time, explore the Galaxy Zoo website and help scientists 
in their exploration of galaxies by performing some simple classifications in their Galaxy Analysis 
section. 

\section*{Homogeneity of galaxy distribution on the sky}

Start Google Earth and click on the ``Saturn'' icon on the top
bar which switches to the Sky portion of Google Earth\footnote{Google Sky is an extension of Google
Earth software into the sky (the simple version of which, Google Maps,
you probably have used to get directions, view a map, etc.).  See the
description of Google Sky that will be distributed to you, as well
as video introduction to Google Sky at {\tt
earth.google.com/sky/skyedu.html}} on the lab computer (or on your
laptop after you install Google Sky and download the SDSS Galaxy Query
kml plugin from the directory showed to you by TA and activate the
\texttt{SDSS Galaxy Photometric Layer} by clicking on it, make sure
the \texttt{SDSS Galaxy Spectroscopic Layer} is turned off (by
unchecking it). Load the kml file {\tt Hubble$\_$lab$\_$random$\_$SDSS$\_$fields.kml}. 

Find the corresponding folder in the {\tt Places} window on the left
hand side of the Sky window and click on $+$ to see the folder
contends ({\tt field 1, field 2, ..., field 10}).  These are ten
random locations in the area of sky covered by the Sloan Digital Sky
Survey away from the plane of the Milky Way and in the areas devoid of
very big bright galaxies and stars (which can block a significant part
of the field of view).  

Click on the {\tt field 1} field and the Sky
will take you to the first field down to the zoom level at which the
field has some approximately 5 minutes of arc size (verify that
the label in the lower right hand
corner of the Sky window shows number close to $0^{\circ}05'$). Please do not zoom
in or zoom out after that. If you do or if you decided to choose your own random field (feel
free to do so),
please zoom again to the point when the label is as close to $0^{\circ}05'$
as possible. This is because we want to compare counts of galaxies
within the same area of the sky in different places, which requires
that the size of the patch we use is the same (5 arcminutes in our
case). Also do not resize the Google Earth window as you count galaxies
in different fields, your counts may be affected if you resize the window. 

Once the Sky finishes its zoom, wait until the SDSS Galaxy Query layer
identifies galaxies brighter than apparent magnitude in $r$ band of $m_r=21$ in the SDSS catalog. 
Once the layer queries the SDSS database over the internet, the galaxies
in the field will be marked by circles. Count the number of marked 
galaxies in the field and then click on {\tt field 2} and repeat the exercise. 

For this part of the lab we need to discuss a little bit of statistics. 
As you know, statistics is a big part of analysis of data in various 
fields: physics, biology, economics, etc. The statistics of counts 
of relatively small random samples of objects is typically governed 
by what is called the Poisson distribution. For example, if galaxies
were distributed completely uniformly on the sky, we would expect that 
the probability to find a given number of galaxies in a field of a given
size is given by the Poisson distribution. The key feature of this 
distribution is that if the average number of galaxies per field of 
a given area is $\bar{N}$, the typical variation from field to field
should be $\sqrt{\bar{N}}$. 

{\bf Lab tasks.}
{\it 
\begin{itemize}
\item Count the number of galaxies brighter than $m_r=21$ (the galaxies
marked by circles) and record the number for each field. 
\item Calculate the average number of galaxies per field using counts from all 
ten fields. You can choose more fields of your own within the areas
covered by the SDSS survey, just make sure that they
are all 5 arcminutes on a side. 
\item Compare counts in each field to the average and expected variation from field
to field. Does it look like the distribution of galaxy counts in these fields
is close to the Poisson distribution (i.e. distribution of galaxies on the sky
is uniform)?
\end{itemize} 
}

\section*{Galaxy Spectra}

In Google Sky activate \texttt{SDSS Galaxy Spectroscopic Layer},
make sure the \texttt{SDSS Galaxy Photometric Layer} is turned off. On the lowest zoom,
find the strips of sky that were observed by the SDSS. Now zoom in and
look for some cool nearby (i.e., bright) galaxies that have spectra available
(circled in brown). Remember, circles are only placed around the
brightest galaxies in your field of view, so if you'd like to check
whether SDSS has a dim galaxy's spectrum, just zoom in on that galaxy
until it is alone or is one of the brightest galaxies.  Find a few
disk/spiral galaxies, and a few elliptical galaxies.  You can
mark galaxies and go back to them later with the \texttt{placemark
tool} (thumbtack in the top toolbar). To view their spectra click on
the desired circled galaxies, and follow the link at the bottom of the
popup bubble.

{\bf Lab tasks.}
{\it \begin{itemize}
\item  Review the spectra of six elliptical and six
spiral galaxies, observe general pattern and similarities in spectra of galaxies of either class.
\item What is the relationship between the morphological class of a galaxy and its spectrum?
 Sketch a typical spectrum characteristic for elliptical and spiral galaxies. 
\item Discuss the main differences in the spectra of elliptical and spiral galaxies and give your 
explanation for the origin of these differences. 
\end{itemize}
}

\section*{Measuring Redshifts}


The gas and stars in all galaxies are made of basically the
same stuff -- the same elements and molecules -- they are just often at different density 
and temperature, with different
abundances from galaxy to galaxy. For example, oxygen we breathe here on Earth is still oxygen in a
distant galaxy, so we can use its properties measured in a lab on Earth to figure out
in what state it is in an observed galaxy. Since each chemical element has a
specific atomic structure, it has a fingerprint of light emission or
absorption that it plants in the spectra of galaxies in the form of emission or absorption spectral lines.
The lines of some chemical species or ions are particularly 
prominent.

 Now that you know what galaxy spectra look like, we will use them to measure
shifts of spectral lines with respect to their laboratory wavelengths. Because these shifts in most galaxies tend 
to be towards longer wavelengths corresponding to redder color, they are called redshifts. You
can find laboratory values for spectral features of elements in the
table below (the wavelengths are given in units of {\AA}ngstr\"oms: $1$\AA=$10^{-8}$~cm, named after the Swedish 
physicist Anders {\AA}ngstr\"om, one of the founders of the field of atomic spectroscopy)

%http://skyserver.sdss.org/en/proj/advanced/hubble/redshifts.asp#ex16 ::
\begin{center}
% use packages: array
\begin{tabular}{lc}
\hline
Element-Transition & Laboratory wavelength (\AA) \\ 
\hline
\textbf{Hydrogen:} & \\
H$\alpha$ & 6563 \\ 
H$\beta$  & 4861 \\ 
H$\gamma$  & 4341 \\ 
H$\delta$  & 4102\\
\hline
\textbf{Heavy elements (metals):} & \\
Mg & 5150 \\ 
Na  & 5892 \\
\hline
%what other elements?
\end{tabular}
\end{center}


Different lines in the spectrum of a given element are indicated in
greek, and roman numerals mark different ionization states, which have
their own distinct spectral fingerprint.

You will measure the redshifts of a set of some of the brightest
galaxies in the universe, which are found in clusters. 
Load file {\tt Hubble$\_$lab$\_$BCGs.kml} into Google Sky and click
on $+$ in front of the folder with the same name in the left window to see 
the list of the brightest cluster galaxies from the list in the Table given
at the end of this document. Click on the first object in the list and Sky 
will take you to the field showing this object. Activate the SDSS Galaxy Spectroscopy
Layer (and switch off the SDSS Galaxy Photometric Layer), the brightest galaxy
should be marked by a circle after the Sky queries the SDSS spectroscopic database. 
Click on the brightest galaxy in the cluster marked by circle and
follow the link to its spectrum in a bubble that pops up.
Determine the redshift of each galaxy by comparing the measured
wavelength of spectral features (absorption or emission lines) with
the wavelength given in the table above as follows 

$$ z={\lambda_{\rm obs}\over\lambda_{\rm lab}}-1$$

Where $\lambda_{\rm obs}$ is the measured wavelength of the spectral
feature and $\lambda_{\rm lab}$ is the wavelength from the ``lab''
measurement from the table. 

{\it 
{\bf Lab tasks.}
\begin{itemize}
\item For each galaxy use two strong absorption lines of magnesium and sodium, labeled Mg and Na. 
Estimate the wavelength of the lines
following the corresponding vertical dotted line to the horizonthal $x$-axis 
and reading off the wavelength number as accurately as you can. 
Please, indicate the lines you used
and provide the actual calculations of redshifts using these lines in your 
lab report. 

\item Each line gives an independent estimate of redshift. Make
sure that they are consistent (close to each other; for example 0.171 and 0.175 are
close but 0.05 and 0.20 are not) and then calculate the average number using 
the two measurements and record this number to the table below.
 The difference between redshifts from individual lines
and the average gives you a rough measure of the error of your redshift 
estimate, which results from the visual estimate of the observed wavelength
$\lambda_{\rm obs}$ from the spectrum. Indicate this error along with the average number. 

\item Click on the galaxy, and a popup bubble should appear containing measurements of light from the galaxy.
 Record the galaxies apparent brightness in the $r$ band (e.g., $r=13.6062$) to the table below. 
The $r$ band is a range of wavelengths in the red part of the spectrum. $r$ band brightness is measured by using
a special filter that filters out light from other wavelengths.

\item Describe the visual appearance of the brightest cluster galaxies and their spectra
and interpret their color and spectra in terms of their stellar populations. 
Do the brightest galaxies vary much in color or shape?
\end{itemize}
}

\section{Making a Hubble Diagram}

The Hubble's law of expansion relates redshift $z=\Delta\lambda/\lambda$ to the distance
between a galaxy and us via a linear relation. The relation is commonly
expressed in terms of the indicative velocity of recession by analogy with the Doppler wavelength
shift we encounter every day in a sound of an ambulance or a passing train,
\begin{equation}
V=cz=c\Delta\lambda/\lambda,
\label{eq:redshift}
\end{equation}
where $c=3\times 10^5$ km/s is the speed of light. The redshifts measured 
for distant galaxies, however, do not actually reflect their physical motion 
in space but rather expansion of the space itself, in which galaxies are markers of the expanding 'net'. 
Nevertheless, the language of velocities is usually adopted for historical reasons and convenience. In these
terms, the recession law is
\begin{equation}
V=H_0d,
\label{eq:hubble}
\end{equation}

where $d$ is the distance to the galaxy and $H_0$ is the constant of
proportionality called the Hubble constant. Since $V$ is commonly measured in
kilometers per second, (km/s) and $d$ is commonly measured in megaparsecs,
(Mpc), the units of $H_0$ must be km/s/Mpc (note that this is equivalent to
units of inverse time). If we know the value of $H_0$ and we have measured the
value of $V$ from the shift of spectral lines, we can use the above formula to
compute the distance. Thus, our perceived scale of the universe depends on the
value of $H_0$. 

Conversely, if we assume that expansion of space have been the same in the past (i.e., the same $V$), 
we can calculate the time it took since the moment when distances between Milky Way
and a galaxy was zero (the moment of the Big Bang) to the present time when the distance is some value $d$: 
$t=d/V$. Given the Hubble law $V=H_0d$, we can rewrite this as $t=d/V=1/H_0$, 
which means that the inverse of the Hubble constant gives us an estimate of the time
elapsed since the Big Bang or the age of the Universe. This is a rough estimate
because modern observations show that $V$ and the Hubble constant are not constant
in time. This estimate however is within a factor of two of the exact age. 

In this part of the lab we will work with data that will allow us to verify the
linear relation between redshift and distance, and to determine the value of
$H_0$. We need some way to measure distances independently of the redshift. We
will use the inverse-square law, where the measured brightness of the whole galaxy
can be related to its distance:
\begin{equation}
f = {L\over 4\pi d^2},
\label{eq:flux}
\end{equation}
where $L$ is the luminosity and is measured in erg/s, $d$ is
the distance measured in cm, and $f$ is apparent brightness,
or flux, $f$, measured in erg/s/cm$^2$.

Different galaxies naturally have different
intrinsic brightnesses (called luminosities in the following). However, Hubble
and others showed that a recognizable class of galaxy, namely the brightest
elliptical galaxies (typically residing in the centers of groups and 
clusters of galaxies), has a small range of luminosity. Such
'first-ranked' galaxies then serve as good ``standard candles'' (objects
with known intrinsic brightness). 

It is often useful to express a physical relation directly in terms of
``observables'' --- quantities we can actually measure in
observations. Equation~\ref{eq:hubble}, called the Hubble law of
recession, contains the apparent recessional velocity $V$, which is
not directly observable, and the distance $d$, which is also not
directly observable.  However, Equations~\ref{eq:redshift} and
\ref{eq:flux} show how to relate these quantities to the redshift $z$
and to the flux $f$ which are directly observable. Write down the
equation that is equivalent to Equation~\ref{eq:hubble} that is
expressed as far as possible in terms of ``observables.''  Your
formula should contain both $L$  and $H_0$. This tells you that an
uncertainty in the value of $H_0$ (the scale of the universe) is
equivalent to an uncertainty in the characteristic luminosities of
galaxies and vice versa.

There is one more formula that is needed. Astronomers like to express
measurements of the flux in terms of magnitudes (see Chapter 1 in the
``Universe'' textbook).  The relation between the magnitude $m$ and
the flux expressed in cgs units is
\begin{equation}
m=-48.60-2.5\log_{10}(f)
\label{eq:mag}
\end{equation}

The numerical factor 48.60 is an arbitrary zero point to the magnitude scale set
by historical precedent. It makes the star Vega ($\alpha$ Lyr) have a magnitude
of 0.0 at a wavelength of 5500~{\AA}. Vega is also used as the zero point for
the B, V, R, I filter system magnitudes. For our purposes, 
the constant factor is not important.

In the last section you compiled a table of redshifts $z$ and apparent
magnitudes for brightest cluster galaxies. We will use the symbol $m_r$ for
r-band magnitudes to avoid confusion. It is useful to extend the Hubble diagram
to very low redshift to increase the lever arm over which we measure $H_0$. 
However, measuring very small redshifts using visual estimates of wavelength used
for the BCGs is tough due to the errors of redshift estimates. Therefore, 
we will rely, as is often done in science, on previous accurate measurement
done by other people for the brighest cluster galaxy NGC 4472 in the nearby Virgo cluster: 
$m_r$ = 9.97 and $z = 0.004$.

 To determine the value of the Hubble constant, $H_0$, we require that one
of the galaxies has a distance that is measured in some fashion that is
independent of the measurement of its redshift (otherwise the argument would be
circular). Such distance measurement is done using distance estimators
from the previous rung of the ``distance ladder'' (e.g., Cepheid stars). 
This is an arduous task, and its description will not be attempted
here. The most recent measurements of the distance to NGC 4472
(the point that should be in the lower left-hand comer of your plot) is 18.2 Mpc.
Given the distance to NGC 4472 and its $r-$band magnitude ($m_r=9.97$), 
you can calculate distances to any other BCG galaxy as (make sure you understand
where this relation comes from; you can derive it from equations given above): 
\begin{equation}
d_{\rm BCG}=18.2\times 10^{0.2(m_r-9.97)}{\ \rm Mpc}. 
\label{eq:dist}
\end{equation}
Using distance measured in such way and the redshift you measured 
for a galaxy, you can can estimate value of $H_0$ for each of the BCGs. 

{\bf Lab tasks.}
{\it 
\begin{enumerate}
\item Make a plot of the data in the accompanying table, plus NGC 4472, that has
log$_{10}z$ on the x-axis and $m_r$ on the y-axis (note that numerically
larger values of $m_r$ correspond to fainter fluxes, {\it i.e.} greater
distances). This plot is called the Hubble diagram. 
From the definition of magnitude in Equation~\ref{eq:mag}, a
difference in flux $f$ of a factor of 100 is equivalent to the difference of 5 magnitudes,
which is what is expected for a difference in
distance of a factor of 10 by the inverse-square law. Since we are claiming that distance is proportional
to redshift, 5 magnitudes should also correspond to a factor of 10 in redshift.
Draw two straight lines through the data points which you think best fit the data 
by eye, one {\em
including} and one {\em excluding} NGC 4472. What is the slope = $\Delta
y/\Delta x$ that you expect if the universe is expanding uniformly? Is that what
you have measured? \label{item:first}

\item Estimate $H_0$ for a dozen BCG galaxies using redshifts you measured and
distances calculated using equation~(\ref{eq:dist}). Average the values to get 
an average estimate of $H_0$. 
What is the derived value for the Hubble constant $H_0$? What is the limit for
the age of the universe that this value implies?

\item Hand in your plot with the value of the Hubble constant clearly indicated.
Include a written discussion of the following questions: Why don't the points
fall on precisely a straight line (give at least two possible reasons)? 
Estimate the rough estimate of time elapsed since the Big Bang using the
value of the Hubble constant you measured ($t=1/H_0$). 
\end{enumerate}
}

%\newpage

%http://umsdss.physics.lsa.umich.edu/catalogs/maxbcg_public_catalog.dat
%$RA,DEC,z,bcgspecz,BCG_rlum,BCG_ilum,rlum_mem,ilum_mem,ngals,ngals_r200$
%$239.58334,27.233419,0.102650,0.0907943,8.67289,10.7264,274.706,343.396,95,188$
%$126.37104,47.133478,0.135050,0.128977,19.0209,24.4289,152.746,192.101,68,99$

%http://xxx.lanl.gov/pdf/astro-ph/0403354v1

\begin{center}
% use packages: array
\begin{tabular}{|c|c|c|c|c|}
\hline
Name & R.A. & dec & z & $m_r$ \\ \hline 
RXCJ0747.0+4131 & 116.7537 & 41.5314 &{\hspace{1cm}} &{\ \ \ \ \ \ \ \ \ \ }\\ \hline  %& 0.0280 \\%13.61
%RXCJ0758.4+3747 & 119.6172 & 37.7888 &{\ \ \ \ \ }&\\ \hline  %& 0.0410 \\ %************% 
RXCJ0810.3+4216 & 122.5942 & 42.2669 &{\ \ \ \ \ }&\\ \hline %& 0.0640 \\ %********%
%RXCJ0041.8-0918 & 10.4587 & -9.3019  &{\ \ \ \ \ }&\\ \hline  %& 0.0520 \\%
RXCJ0809.6+3455 & 122.4177 & 34.9262 &{\ \ \ \ \  }&\\ \hline % & 0.0800 \\ %*******% 
RXCJ0800.9+3602 & 120.2445 & 36.0469  &{\ \ \ \ \  }&\\ \hline % & 0.2880 \\%
%RXCJ0114.9+0022 & 18.7350 & 0.3746   & &\\  %& 0.0470 \\%this is off center!%
%RXCJ0119.6+1453 & 19.9072 & 14.8931  & &\\ %& 0.1290 \\%also hard to tell%
%RXCJ0137.2-0912 & 24.3140 & -9.2028   %& 0.0390 \\ too hard to tell%
%RXCJ0152.7+0100 & 28.1762 & 1.0126   & &\\ \hline  %& 0.2270 \\%
RXCJ0736.4+3925 & 114.1540 & 39.4229  & &\\ \hline  % 114.1040 & 39.4329  %& 0.1170 \\%
%RXCJ0753.3+2922 & 118.3291 & 29.3741 & 0.0620 \\ too hard to tell%
%%%%%%
%not in DR5? RXCJ0821.8+0112 & 125.4655 & 1.2116 & 0.0820 &  \\ \hline%
RXCJ0822.1+4705 & 125.5417 & 47.0995 & &\\ \hline  % 0.1300 & 15.52\\ \hline%15.52
%RXCJ0824.0+0326 & 126.0209 & 3.4383 & 0.3470 & 18.35\\ \hline %BCG location?
RXCJ0825.4+4707 & 126.3652 & 47.1196 & &\\ \hline  % 0.1260 & 14.71\\ \hline %***
RXCJ0828.1+4445 & 127.0278 & 44.7634 & &\\ \hline  % 0.1450 & 15.71\\ \hline %nicenodata
RXCJ0842.9+3621 & 130.7401 & 36.3625 & &\\ \hline  % 0.2820 & 16.79\\ \hline %
%RXCJ0845.3+4430 & 131.3434 & 44.5115 & &\\ \hline  % 0.0540 & 14.82\\ \hline% weak
%RXCJ0850.1+3603 & 132.5499 & 36.0614 & &\\ \hline  % 0.3730 & 18.57\\ \hline%
RXCJ0913.7+4056 & 138.4411 & 40.9339 & &\\ \hline  % 0.4420 & 18.14\\ \hline% crazy high z
RXCJ0913.7+4742 & 138.4446 & 47.7021 & &\\ \hline  % 0.0510 & 14.52\\ \hline%  ***
RXCJ0917.8+5143 & 139.4637 & 51.7223 & &\\ \hline  % 0.2170 & 16.73\\ \hline% * super rich but only 3 specs
RXCJ0943.0+4700 & 145.7600 & 47.0038 & &\\ \hline  % 0.4060 & 19.00\\ \hline%
%RXCJ0947.1+5428 & 146.7862 & 54.4754 & &\\ \hline  % 0.0460 & 14.17\\ \hline%
RXCJ0952.8+5153 & 148.2009 & 51.8888 & &\\ \hline  % 0.2140 & 16.56\\ \hline%
RXCJ0953.6+0142 & 148.4231 & 1.7118  & &\\ \hline  % 0.0980 & 14.82\\ \hline%
RXCJ1000.5+4409 & 150.1260 & 44.1550 & &\\ \hline  % 0.1540 & 17.15\\ \hline%
%RXCJ1013.7-0006 & 153.4368 & -0.1085 & &\\ \hline  % 0.0930 & 14.76\\ \hline%** but no bcg 
%*RXCJ1017.5+5933 & 154.3960 & 59.5577 & &\\ \hline  % 0.2850& 17.76\\ \hline%corrected z ** but no bcg
%RXCJ1022.5+5006 & 155.6283 & 50.1030 & &\\ \hline  % 0.1580 & 15.51\\ \hline%
%RXCJ1023.6+0411 & 155.9125 & 4.1873 & &\\ \hline  % 0.2850  & 17.17\\ \hline%
%RXCJ1023.6+4908 & 155.9212 & 49.1349 & &\\ \hline  % 0.1440 & 15.45\\ \hline%***
RXCJ1053.7+5452 & 163.4349 & 54.8726 & &\\ \hline  % 0.0750 & 15.24\\ \hline%
RXCJ1058.4+5647 & 164.6097 & 56.7922 & &\\ \hline  % 0.1360 & 15.48\\       %***
\hline
\end{tabular}
\end{center}

%\includegraphics[width=7in]{g0312-220.pdf}
%\newpage
%\includegraphics[width=7in]{g0312-255.pdf}
%\newpage
%\includegraphics[width=7in]{g0312-529.pdf}
%\newpage
%\includegraphics[width=7in,trim=1.in 0.in 0.3in 0.2in]{Template_spect2.pdf}
%\newpage
%\begin{center}
%
%\includegraphics[width=5.5in,trim=2.8in 0in -2.in 2.3in]{data.pdf}\\

%\end{center}


\end{document}
