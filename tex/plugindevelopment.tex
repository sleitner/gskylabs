\documentclass[12pt]{article}
%\usepackage{mipsections}
\usepackage{graphicx}
\usepackage{dcolumn}
\usepackage{array}
\setlength{\textwidth}{6.5 in}
\setlength{\oddsidemargin}{0pt}
\setlength{\evensidemargin}{0pt}
\setlength{\textheight}{9 in}
\newlength{\newheadsep}
\setlength{\newheadsep}{0.5in}
\setlength{\headheight}{\normalbaselineskip}
\addtolength{\newheadsep}{-\headheight}
\setlength{\headsep}{\newheadsep}
\setlength{\topmargin}{-0.5in}


\begin{document}
\vspace*{-.7in}
\rightline{\tiny PS12000 Winter 2003 V1.1}
\vspace{.7in}
\centerline{\large\bf Manual for Plugin Modification and Development}
\vspace{.2in}




Adding any simple layer to Sky requires several lines of markup code so scripts that loop these changes are necessary. Fortunately, Sky efficiently loads layer data and up to tens of mb of KML should not be a problem. The easiest way to understand how to use KML and how to write the necessary scripts is to jump in with examples. In this section I will explain components of the scripts in sufficient detail so that they can be easily understood and modified. 
\section*{Querying the SDSS}

\section*{Graphing Quantities}

\end{document}
