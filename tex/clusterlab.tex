\documentclass[12pt]{article}
%\usepackage{mipsections}
\usepackage{graphicx}
\usepackage{dcolumn}
\usepackage{array}
\setlength{\textwidth}{6.5 in}
\setlength{\oddsidemargin}{0pt}
\setlength{\evensidemargin}{0pt}
\setlength{\textheight}{9 in}
\newlength{\newheadsep}
\setlength{\newheadsep}{0.5in}
\setlength{\headheight}{\normalbaselineskip}
\addtolength{\newheadsep}{-\headheight}
\setlength{\headsep}{\newheadsep}
\setlength{\topmargin}{-0.5in}


\begin{document}
\vspace*{-.7in}
\rightline{\tiny PS12000 Winter 2009 V1.0}
\vspace{.7in}
\centerline{\large\bf Physci 120, Winter 2009}\vspace{0.4cm}
\centerline{\Large\bf Lab 2}\vspace{0.4cm}
\centerline{\Large\it  Galaxy environments, clusters of galaxies, and dark matter}
\vspace{.2in}

\section*{Introduction}

Although the distribution of galaxies on large scales is quite uniform, 
on smaller scales, comparable to the mean separation between bright galaxies
(about one million parsecs), galaxy distribution is far from being homogeneous. 
Galaxies are arranged in a variety of large {\it structures\/} from 
 {\it groups\/} of a few galaxies (the nearest example is {\it the Local Group} formed by our Milky Way, Andromeda Galaxy (M31) and a few dozen faint
satellite galaxies, including the Magellan Clouds) and 
{\it clusters of galaxies\/} containing dozens and hundreds of bright galaxies,
to galaxy superclusters and filaments which span tens of millions of parsecs
in size. 

Although these structures can be clearly seen in the diagrams
showing three-dimensional distribution of galaxies, in most cases 
they are not as obvious in the distribution of galaxies in the sky. 
This is because on the sky we see galaxies at different distances projected
onto the same area, which tends to ``wash out'' the structures. 
Nevertheless, the basic tendency of galaxies to cluster can be estimated
by counting numbers of galaxy pairs separated by a certain angle on the sky
and comparing it to the number of pairs expected for a uniform distribution. 
Such statistic, called {\it galaxy correlation function}, has historically 
been a very powerful tool of studying clustering of galaxies and is still actively
used in astronomy. 

In the first part of the lab, you will perform a simple counts
exercise to see the tendency of galaxies to cluster for yourself. You
will then explore environments of typical galaxies in random fields
(called {\it field galaxies}) and galaxies in groups and clusters. The
key lesson of this exercise will be that galaxies in clusters are
systematically different than galaxies in random fields: namely,
clusters contain unusual concentrations of bright galaxies and a
most galaxies in a cluster have similar colors and are redder than typical 
galaxies in the field . You will see that this fact can be used to search
for clusters on the sky using galaxy colors and you will carry out
such search based on what you learn in the first half of the lab.

In the final part of the lab you will estimate total gravitational 
mass of a cluster using the virial theorem and compare it to the 
total mass associated with its galaxies. You will see that the two masses
are vastly different, thereby recovering one of the oldest and strongest 
observational evidence for the existence of {\it dark matter} in the universe, 
which was originally obtained by US astrophysicists 
Fritz Zwicky and Sinclair Smith in the mid-1930s using very similar method to what 
you will use. 

\section*{Galaxy clustering}

As noted above, galaxies tend to {\it cluster} on the scales of the order
of mean separation between galaxies. We can quantify this tendency by 
comparing the number of galaxies within a certain angular radius of 
another galaxy and compare it to the number expected if galaxies
were distributed uniformly on the sky. 

For this exersize we can use the random fields used in the Hubble lab
to perform galaxy counts reuse your previous counts or you can choose a random field within the 
sky area covered by the SDSS survey yourself. Load the SDSS Galaxy Query
kml plugin from the directory showed to you by TA and activate the
\texttt{SDSS Galaxy Photometric Layer} by checking the box next to it, make sure
the \texttt{SDSS Galaxy Spectroscopic Layer} is turned off (by
unchecking it). Zoom in on a random field of 5 arcminutes accross (the size
is indicated in the lower right corner of the Sky window). 

Count galaxies within the fields identified by the photometric plugin (galaxies 
with circles). To estimate the mean separation between galaxies on the sky, 
we need to estimate their mean density: $\bar{n}=N/A$, where $N$ is the number of 
galaxies in the area and $A$ is the area, which we will compute in arcminutes
squared (arcmin$^2$). To measure the area of the viewing window, use the ruler 
icon in the icon bar at the top of the Sky window. Once you click on the ruler
icon, a window and a target box will pop up. Bring cursor to one edge of the viewing
area and click, then bring the cursor to the opposite edge (you will see the line 
stretching after the cursor, make sure the line is horizonthal or vertical, depending
on which side of the viewing rectangle you are measuring) then click when the cursor
reaches the edge. The ruler window will report the angular distance between the two 
points (i.e., the length of the stretched line) in degrees (recall that 1 degree is
equal to 60 arcminutes). Compute the area of the viewing window rectangle in arcminutes
squared using
your measurements and the mean density of galaxies in your area. You can reuse
the counts from the previous lab for other fields to estimate the mean number
of galaxies better. \\[2mm]


{\bf Lab tasks.} 
{\it
\begin{itemize}
\item Estimate the mean density of galaxies with SDSS photometry, $\bar{n}$, (identified by circles
by the photometric plugin). Report the number and the details of calculations used to 
obtain it. 
\item Estimate the mean angular distance between galaxies used in the counts
by taking an inverse cube root of the mean density: $\bar{d}=\bar{n}^{-1/3}$.
Report this number as well. 
\item Choose one of the galaxies used in estimating of $\bar{d}$ and 
zoom-in on it to the zoom level (number in right bottom corner) equal 
to $2\bar{d}$ (be careful to estimate the number in the same units, the scale in the
corner is given in arcminutes and arcseconds). Count the number of 
the circled galaxies within the regions. Then zoom-out to the original view, 
choose another galaxy and repeat. Perform such counts for a dozen galaxies or
so. 
\item How many galaxies do you usually find when you zoom-in to $2\bar{d}$?
Is this consistent with what you expect for a uniform distribution?  
\end{itemize}
}

\section*{Galaxy luminosities and colors in different environments}

We will first explore galaxies in the typical random field (called ``field galaxies''). 
Make sure the \texttt{SDSS Galaxy Query plugin} used in the previous part is still
activated and it identifies galaxies with circles, and activate the \texttt{Color-Magnitude plugin}, which plots the $r$-band apparent magnitude ($x$-axis) and colors (difference
in apparent magnitude in $g$ and $r$ filters, in the $y$-axis) of galaxies in the field
in the graph in the top left corner of the field of view. This plot is called {\it the color-magnitude diagram\/} (CMD). 
You can use one of the random fields used before, or choose your own. Zoom on the field
to the scale of about 15 arcminutes or so (number in the lower right corner).\\[2mm] 

{\bf Lab tasks.} 
{\it
\begin{itemize}
\item Describe how galaxies in the random fields are distributed in the CMD. 
\item Load {\tt Groups} location file and examine CMD around galaxy groups (small groupings
of galaxies. Describe any difference you notice from the CMD distribution of galaxies in
the random fields. 
\item Load {\tt Clusters} location file and describe the distribution of galaxies in the CMDs
and the differences from random and group locations. 
\item Zoom-in onto the region around the brightest cluster galaxies (say central 5 arcminutes) and
examine the CMD of galaxies in this region. Now zoom-out, center on a region in the outskirts 
of the cluster (say 20-30 arcminutes away from the BCG), zoom-in to 5 arcminute level again and examine
CMD. How does the distribution of galaxies changes when you move to the
outskirts of clusters?
\item Examine clusters at different redshifts (you can get the redshift by clicking on the
brightest galaxy near the center of each cluster). How does the distribution of galaxies 
change in the CMD for clusters at higher redshifts?
\item Discuss and interpret your results in terms of what you learned about properties
of galaxies in nearby clusters and what you know about spectra of red elliptical galaxies. 
\end{itemize}
}

\section*{Hunting for clusters}

In the previous section you should have learned that cluster fields
contain unusual concentration of bright galaxies which look more
yellow (``redder'') than most field galaxies. Galaxies in clusters
form a horizonthal ``ridge'' or ``sequence'' in the color-magnitude
diagram. In this portion of the lab you will use this fact to carry
out a ``blind'' search for galaxy clusters in the sky. This method of
searching for clusters, done in an automated and systematic way using
a computer analysis of images, was pioneered just ten years ago by the
University of Chicago Prof. Michael Gladders (among others) and is the
most efficient way of finding new clusters using the optical images
and multi-band photometry (measurements of apparent magnitudes of
galaxies through different filters).

To perform the search, go to a random field on the sky (you can go to one
of the random fields used above or choose a random area on the sky yourself, 
just make sure you are well within the SDSS coverage). Zoom-in (or zoom-out
depending on the inital zoom of your view) to the level 
at which the marker in the lower right hand corner of the Sky view window
shows approximately 15 arcminutes (i.e., $0^{\circ}15^{'}...$). 

Once you choose
the zoom level, keep it the same throughout the exercise (you can play with it
once you find a cluster). 
Now turn off (uncheck) the \texttt{imagery layer} in the bottom list
on the panel on the left hand side of the Sky window.  This should get
rid of the SDSS image of the sky, leaving a gray uniform background
with the CMD diagram remaining. Now scan the sky making $\approx
15^{'}$ steps.  You can do this, for example, by clicking on the left
most edge of the field of view and dragging it all the way to the
right (or, similarly, from the bottom to top). The order or direction of your steps
is not very important, but it may make sense to do them in one direction once you chose
it. Please count the number of steps you take (thereby roughly estimating the
area you scan to find a cluster). 

If you stumble onto an area which may credibly look like it contains a
cluster, you can try to make smaller shifts about the center of the
field of view to see if it makes cluster signature in the CMD more
pronounced. If you believe you may have found a cluster, turn the
\texttt{imagery} layer back on and scan the field of view visually. If
it looks like there is a cluster in particular area, zoom on it and
examine to make sure (hint: look for large, bright elliptical galaxies
in the field of view similar to the brightest cluster galaxies you've examined in the
Hubble lab; these are candidates for cluster centers which you can examine by 
zooming in on their locations). If you are not sure there is a cluster after visual
inspection, consult with the TA, or simply continue searching for a
more obvious case. Once you find the cluster, you are ready for\\[2mm]

{\bf Lab tasks.} 
{\it
\begin{itemize}
\item Estimate the rough area you surveyed to find a cluster. How many such clusters per square degree of the sky would you then expect? How rare are clusters based on your measurements (how many would there by in the entire sky; the entire sky has the area of approximately $41253$ square degrees)?
\item Turn on the {\tt SDSS Galaxy Query} layer (uncheck the {\tt Photometric Layer}, make
sure the {\tt Spectroscopic layer} is turned on. Find galaxies with SDSS spectra, look for
the spectra of the brightest galaxies in the cluster area. Find the approximate redshift
of the cluster. 
\item Examine galaxies with SDSS spectra in the area. Do all of them have similar redshifts. 
If not all, why not?
\item Turn of the CMD layer, zoom-out to view larger than $15^{'}$ accross. Examine
redshifts of galaxies with SDSS at larger distances from the brightest cluster galaxies. 
Try to find the most distant galaxy from the BCG which still has redshift similar to that
of the BCG. Measure the angular distance between BCG and the most distant galaxy using the ruler
icon. Quote the number --- this will be your rough estimate of the cluster extent. 
\item Identify several (the more, the better but not more than 10) 
galaxies within extent of the cluster with redshifts close to that
of the BCG. Record their redshifts and calculate the mean redshift of the cluster using
their individual redshifts. 
\item Using the angular extent measured above and the estimated redshift of the cluster, 
compute the actual physical extent of the cluster in Mpc corresponding to the angular extent
assuming the Hubble constant value of $H_0=70$~km/s/Mpc. 
\end{itemize}
}

\section*{Measuring cluster mass using Virial Theorem}

If, in the previous section of the lab, you found a nearby cluster ($z\sim 0.0.05-0.15$) with 
at least five member galaxies that had spectra and similar redshifts to that of the BCG, you can use
the cluster you found to measure its mass using the virial theorem in this section. If the cluster
you found is at higher redshift and/or does not have enough galaxies with measured spectra, you can
use one of the systems in the {\tt Clusters} list used above to examine cluster environments. 
In the latter case, you have to redo the exercise of estimating cluster extent on the sky and
converting it into physical extent in megaparsecs. 

Once the cluster is chosen, center the field of view on the BCG and adjust zoom so that the field of view
encompasses the entire extent of the cluster. Identify cluster members (i.e., galaxies with SDSS spectra and
measured redshifts similar
to that of the BCG) among galaxies identified with a circle with the SDSS Galaxy Spectroscopic plugin. 
Record the redshifts of the member galaxies and calculate the mean redshift of the cluster. 

To estimate cluster mass, you will use the so-called {\it virial theorem}. This theorem 
relates the kinetic, $K$, and potential, $W$, energies of the system and applies to any self-gravitating system 
in equilibrium (e.g. it also applies to the Sun, Earth, stellar clusters, etc.): 
\begin{equation}
W=-2K.
\label{eq:VT}
\end{equation}
The kinetic energy of a cluster is 
\begin{equation}
K\approx \frac{1}{2}M_{\rm tot}\langle v_{\rm gal}^2 \rangle, 
\label{eq:KE}
\end{equation}
where $M_{\rm tot}$ is the {\it total} gravitating mass of a cluster (which we aim to measure) 
and $\langle v_{\rm gal}^2\rangle$ is characteristic velocity disperson of cluster material. We 
will estimate the latter by using dispersion of velocities of the cluster member galaxies about the 
mean cluster velocity, which can be calculated as
\begin{equation}
\langle v^2_{\rm gal} \rangle \equiv 3\,{\sum_{i=1}^{N_{\rm gal}} (cz_i-c\bar{z})^2\over N-1}= 3c^2\,\times\,{\sum_{i=1}^{N_{\rm gal}} (z_i)^2 - N_{\rm gal}\bar{z}^2\over N_{\rm gal}-1}. 
\label{eq:disp}
\end{equation}

The factor of 3 in front is to account for the fact that redshifts measure only velocity
along the line of sight, while virial theorem applies to velocities along all three dimensions of space; 
$\bar{z}$ is the mean redshift of the cluster you measured using the $N_{\rm gal}$ galaxies with
spectroscopic redshifts $z_i$, which you have recorded in the previous step. 

The potential energy of the self-gravitating system of mass $M_{\rm tot}$ is negative by definition and is given by
\begin{equation}
W=-\alpha \frac{GM_{\rm tot}^2}{R},
\label{eq:PE}
\end{equation}
where $\alpha$ is a numeric factor of order unity which depends on how mass is distributed 
within the system, $R$ is the physical extent of the system, and $G=6.6726\times 10^{-8}\,\,\rm cm^3\, s^{-2}\,g^{-1}$ (in cgs units) is the universal gravitational
constant. For the mass distribution expected in clusters $\alpha\approx 0.5$ (good assumption
for the accuracy of your measurement which will be to a factor of two or so), and 
we will use this number to compute the mass. 

Combining equations~\ref{eq:VT}, \ref{eq:KE}, and \ref{eq:PE} we get: 
\begin{equation}
M_{\rm tot}=\frac{R\langle v^2_{\rm gal} \rangle}{\alpha G}.
\label{eq:M}
\end{equation}
\\[2mm]

{\bf Lab tasks:} 

{\it 
\begin{itemize}
\item Using your tabulated redshifts $z_{i}$ of galaxy members calculate the mean redshift $\bar{z}$ 
and $\langle v^2_{\rm gal}\rangle$ in km/s. 
\item Use equation~\ref{eq:M} to estimate total mass of the cluster within the extent you measured. 
$R$ is half the cluster diameter that you measured and take $\alpha=0.5$. Be careful with units and 
give the result in solar masses $M_{\odot}$ (where $M_{\odot}=1.989\times 10^{33}$~g). 
\item Count the number of red bright galaxies around the cluster within $R$ (with or without spectra) 
to get a rough number of cluster galaxies. Each of such bright galaxies has total mass in stars
of about $10^{11}\,\rm M_{\odot}$ (give or take a factor of 2-3). This number is determined
from luminosities of these galaxies and average luminosity and mass of old stars
which these galaxies consist of. Estimate roughly the total mass
in stars within $R$ using the number of galaxies you counted and the average stellar mass
of $10^{11}\,\rm M_{\odot}$. How does this mass compare to the total mass $M_{\rm tot}$ you calculated
using the virial theorem? How big is the difference?
\end{itemize}
}

The difference between mass in stars and total mass is one of the pieces of observational
evidence we have for existence of large amounts of dark gravitating matter in the universe on large scales. 
Actually, in clusters there are large amounts of hot plasma mostly consisting of hydrogen and helium
nuclei and electrons, but the mass of this plasma measured from its X-ray emission still falls far short
of explaining the measurements of total cluster masses. 

\end{document}

